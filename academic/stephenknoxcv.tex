%%%%%%%%%%%%%%%%%%%%%%%%%%%%%%%%%%%%%%%%%%%%%%%%%%%%%%%%%%%%%%%%%%%%%%%%
%%%%%%%%%%%%%%%%%%%%%% Simple LaTeX CV Template %%%%%%%%%%%%%%%%%%%%%%%%
%%%%%%%%%%%%%%%%%%%%%%%%%%%%%%%%%%%%%%%%%%%%%%%%%%%%%%%%%%%%%%%%%%%%%%%%

%%%%%%%%%%%%%%%%%%%%%%%%%%%%%%%%%%%%%%%%%%%%%%%%%%%%%%%%%%%%%%%%%%%%%%%%
%% NOTE: If you find that it says                                     %%
%%                                                                    %%
%%                           1 of ??                                  %%
%%                                                                    %%
%% at the bottom of your first page, this means that the AUX file     %%
%% was not available when you ran LaTeX on this source. Simply RERUN  %% 
%% LaTeX to get the ``??'' replaced with the number of the last page  %% 
%% of the document. The AUX file will be generated on the first run   %%
%% of LaTeX and used on the second run to fill in all of the          %%
%% references.                                                        %%
%%%%%%%%%%%%%%%%%%%%%%%%%%%%%%%%%%%%%%%%%%%%%%%%%%%%%%%%%%%%%%%%%%%%%%%%

%%%%%%%%%%%%%%%%%%%%%%%%%%%% Document Setup %%%%%%%%%%%%%%%%%%%%%%%%%%%%

% Don't like 10pt? Try 11pt or 12pt 
\documentclass[11pt]{article} 

% This is a helpful package that puts math inside length specifications
\usepackage{calc}

% Layout: Puts the section titles on left side of page
\reversemarginpar

%
%         PAPER SIZE, PAGE NUMBER, AND DOCUMENT LAYOUT NOTES:
%
% The next \usepackage line changes the layout for CV style section
% headings as marginal notes. It also sets up the paper size as either
% letter or A4. By default, letter was used. If A4 paper is desired,
% comment out the letterpaper lines and uncomment the a4paper lines.
%
% As you can see, the margin widths and section title widths can be
% easily adjusted.
%
% ALSO: Notice that the includefoot option can be commented OUT in order
% to put the PAGE NUMBER *IN* the bottom margin. This will make the
% effective text area larger.
%
% IF YOU WISH TO REMOVE THE ``of LASTPAGE'' next to each page number,
% see the note about the +LP and -LP lines below. Comment out the +LP
% and uncomment the -LP.
%
% IF YOU WISH TO REMOVE PAGE NUMBERS, be sure that the includefoot line
% is uncommented and ALSO uncomment the \pagestyle{empty} a few lines
% below.
%

%% Use these lines for letter-sized paper
\usepackage[paper=letterpaper,
            %includefoot, % Uncomment to put page number above margin
            marginparwidth=1.2in,     % Length of section titles
            marginparsep=.05in,       % Space between titles and text
            margin=1in,               % 1 inch margins
            includemp]{geometry}

%% Use these lines for A4-sized paper
%\usepackage[paper=a4paper,
%            %includefoot, % Uncomment to put page number above margin
%            marginparwidth=30.5mm,    % Length of section titles
%            marginparsep=1.5mm,       % Space between titles and text
%            margin=25mm,              % 25mm margins
%            includemp]{geometry}

%% More layout: Get rid of indenting throughout entire document
\setlength{\parindent}{0in}

%% This gives us fun enumeration environments. compactitem will be nice.
\usepackage{paralist}

%% Reference the last page in the page number
%
% NOTE: comment the +LP line and uncomment the -LP line to have page
%       numbers without the ``of ##'' last page reference)
%
% NOTE: uncomment the \pagestyle{empty} line to get rid of all page
%       numbers (make sure includefoot is commented out above)
%
\usepackage{fancyhdr,lastpage}
\pagestyle{fancy}
%\pagestyle{empty}      % Uncomment this to get rid of page numbers
\fancyhf{}\renewcommand{\headrulewidth}{0pt}
\fancyfootoffset{\marginparsep+\marginparwidth}
\newlength{\footpageshift}
\setlength{\footpageshift}
          {0.5\textwidth+0.5\marginparsep+0.5\marginparwidth-2in}
\lfoot{\hspace{\footpageshift}%
       \parbox{4in}{\, \hfill %
                    \arabic{page} of \protect\pageref*{LastPage} % +LP
%                    \arabic{page}                               % -LP
                    \hfill \,}}

% Finally, give us PDF bookmarks
\usepackage{color,hyperref}
\definecolor{darkblue}{rgb}{0.0,0.0,0.3}
\hypersetup{colorlinks,breaklinks,
            linkcolor=darkblue,urlcolor=darkblue,
            anchorcolor=darkblue,citecolor=darkblue}

%%%%%%%%%%%%%%%%%%%%%%%% End Document Setup %%%%%%%%%%%%%%%%%%%%%%%%%%%%


%%%%%%%%%%%%%%%%%%%%%%%%%%% Helper Commands %%%%%%%%%%%%%%%%%%%%%%%%%%%%

% The title (name) with a horizontal rule under it
%
% Usage: \makeheading{name}
%
% Place at top of document. It should be the first thing.
\newcommand{\makeheading}[1]%
        {\hspace*{-\marginparsep minus \marginparwidth}%
         \begin{minipage}[t]{\textwidth+\marginparwidth+\marginparsep}%
                {\large \bfseries #1}\\[-0.15\baselineskip]%
                 \rule{\columnwidth}{1pt}%
         \end{minipage}}

% The section headings
%
% Usage: \section{section name}
%
% Follow this section IMMEDIATELY with the first line of the section
% text. Do not put whitespace in between. That is, do this:
%
%       \section{My Information}
%       Here is my information.
%
% and NOT this:
%
%       \section{My Information}
%
%       Here is my information.
%
% Otherwise the top of the section header will not line up with the top
% of the section. Of course, using a single comment character (%) on
% empty lines allows for the function of the first example with the
% readability of the second example.
\renewcommand{\section}[2]%
        {\pagebreak[2]\vspace{1.3\baselineskip}%
         \phantomsection\addcontentsline{toc}{section}{#1}%
         \hspace{0in}%
         \marginpar{
          \raggedright \scshape #1}#2}

% An itemize-style list with lots of space between items
\newenvironment{outerlist}[1][\enskip\textbullet]%
        {\begin{itemize}[#1]}{\end{itemize}%
         \vspace{-.6\baselineskip}}

% An environment IDENTICAL to outerlist that has better pre-list spacing
% when used as the first thing in a \section 
\newenvironment{lonelist}[1][\enskip\textbullet]%
        {\vspace{-\baselineskip}\begin{list}{#1}{%
        \setlength{\partopsep}{0pt}%
        \setlength{\topsep}{0pt}}}
        {\end{list}\vspace{-.6\baselineskip}}

% An itemize-style list with little space between items
\newenvironment{innerlist}[1][\enskip\textbullet]%
        {\begin{compactitem}[#1]}{\end{compactitem}}

% To add some paragraph space between lines.
% This also tells LaTeX to preferably break a page on one of these gaps
% if there is a needed pagebreak nearby.
\newcommand{\blankline}{\quad\pagebreak[2]}

%%%%%%%%%%%%%%%%%%%%%%%% End Helper Commands %%%%%%%%%%%%%%%%%%%%%%%%%%%

%%%%%%%%%%%%%%%%%%%%%%%%% Begin CV Document %%%%%%%%%%%%%%%%%%%%%%%%%%%%

\begin{document}
\makeheading{Stephen Knox \small{PhD} \small{BA (Mod)}}

\section{Contact Information}
%
% NOTE: Mind where the & separators and \\ breaks are in the following
%       table.
%
% ALSO: \rcollength is the width of the right column of the table 
%       (adjust it to your liking; default is 1.85in).
%
\newlength{\rcollength}\setlength{\rcollength}{2.85in}%
%
\begin{tabular}[t]{@{}p{\textwidth-\rcollength}p{\rcollength}}
60 Broughton Court,       & \textit{T (Mob):} (+44) 77 8699 8383\\               
Broughton Road,			      & \textit{T (Home):} (+44) 208 621 8844\\
London W13 8QN         & \textit{E:}
\href{mailto:knoxsp@gmail.com}{knoxsp@gmail.com}\\
England           & \\
\end{tabular}

\section{Personal Statement}

With a PhD in computer science and over five years experience in industrial grade
software development, I am now looking for a new challenge. Over the last five
years, I have developed full-stack web solutions for gaming websites and an
open-source data management system for environmental modelling. I am now
looking to build on this experience in a new, client-facing role focussed on delivering high
quality software. Strongly team-focussed, I am eagre to work on exciting new projects,
and working with cutting-edge technologies and development methodologies.

%\section{Research Interests}
% Research interests are as follows:
%\begin{itemize}
%\item Integrating software engineering into research practices.
%\vspace{-3 mm}
%\item Simulation of large-scale systems such as water resource networks or pervasive systems.
%\vspace{-3 mm}
%\item Engineered resource network management and planning.
%\vspace{-3 mm}
%\item Reasoning systems within Pervasive Computing. Highly experienced in case-based reasoners, rule based reasoners and Bayes networks.
%\vspace{-3 mm}
%\item Evaluation of pervasive systems with respect to data acquisition, representation and reasoning.
%\vspace{-3 mm}
%\item Middleware for pervasive systems.
%\vspace{-3 mm}
%\item Context-Aware computing applications.
%\vspace{-3 mm}
%\end{itemize}

\section{Personal Skills}
Highly focussed and an excellent communicator, giving me the
ability to work autonomously and in a team environment. \\
\\
Extensive experience in research and development. \\
\\
Highly effective and experienced communicator at all levels and to
varied audiences.\\
\\
Experienced in several programming languages and techniques. Can easily adapt
to new challenges.

\section{Technial Skills}
I have a strong background in computer science and therefore can quickly adapt to most new technologies
and languages. I am passionate about well-documented, tested and robust code and
the development of open-source software.
\\
\\
 - \textbf{Python} (incl. PyUnit, sqlalchemy, turbogears, numpy, pandas, scipy) \textgreater 8 years
\\
 - \textbf{Databases} (incl. Sqlite, MySQL, Oracle) \textgreater 5 years
\\
 - Web Client Technologies: \textbf(Javasvript) (incl. jquery, visjs, d3, datatables, bootstrap), \textbf{CSS3}, \textbf{HTML5} \textgreater 5 years
\\
 - \textbf{Web Apis}: SOAP, REST, JSON Rpcs, \textgreater 5 years
\\
 - Version Control: \textbf{GIT}, \textbf{SVN} \textgreater 8 years
\\
 - \textbf{Java} (incl. JUnit, Jena, Eclipse): 3 years. Proficient
\\
 - \textbf{C\#}: Proficient
\\
 - Publishing: Microsoft Office, Open Office, Latex, Lucidchart
\\
 - Using \textbf{Agile} metholdogies with \textbf{Jira} and \textbf{trac}
\\
- OS: \textbf{Ubuntu} daily for \textgreater 8 years. \textbf{Vim} for text editing, 

\newpage

\section{Education}
\textbf{2005 to 2010:} PhD in Computer Science, UCD Dublin.\\
PhD entitled ``Combining Case-Based Reasoning and the Semantic Web in
Recognising Situations''. Viva passed 1st April, 2010. Built an Java-based artificial intelligence
system to identify the actions of people in a sensor-rich environment. Worked extensively with
real-time sensor technologies, Java, Protege Ongology editor and Weka.

 \textbf{2001 to 2005:} \textit{2.1}, BA (Mod) (Hons),
Computer Science, Trinity College, Dublin, Ireland.

\textbf{1996 to 2001:} 475 points from 600, Leaving Certificate, St
Kieran's College, Kilkenny, Ireland. Leaving certificate results: Maths: B1, Economics B1, English B3, 
Irish C1, Spanish B3,Chemistry B2, Music B3.

\section{Work Experience}
\textbf{July 2013 to Present}: Senior scientific software engineer \& team lead, University of Manchester.
Developing an open source tool, Hydra Platform, for managing network-based data in water resoure planning.
Led the design and implementation a web API, requiring expert knowledge of Python, databases
(MySql, Sqlite, PostreSQL) and web interfaces (REST, SOAP, JSON, HTML, CSS \& Javascript).
Developed several client applications which connect to Hydra Platform using Python, Java, C\# \& Javascript.

In collaboration with project partner Thames Water Utility Limited (TWUL),
leading the development of a web-based 3D trade-off visualisation tool. Implemented
using visjs, turbogears, sqlalchemy html 5, css3. 
\\
\\
\textbf{April 2010 to June 2013}: Software developer, Geneity Ltd.
Developed production quality software for large clients, most notably
Coral, Ladbrokes and The Health Lottery. These websites can take up to 200 bets a second,
meaning a high standard of coding, testing and support was required.
Supported these products in production environment. Also heavily involved in developing the back-end
administration for these websites. Worked extensively with web technologies such
as HTML5, CSS3, Javascript including jQuery, Oracle and Python.

Responsibilities included prototyping, developing and deploying products
to test and live environment. Dealt extensively with clients both during
development and testing process.\\

\textbf{January 2006 to April 2010}: Lectured in the principles of programming languages, specifically on compiler
design and virtual machines.\\
Teaching assistant in UCD Computer Science dept. Attended labs and
marked assignments and performed various administrative tasks.
Senior member of the Computer Programming Support Centre. Roles include tutoring
students on Java, C++, Scheme and related logic problems.\\

\textbf{Consultency 2008 -- 2009}: Recipient of an Enterprise Ireland Innovation Voucher
\footnote{http://www.innovationvouchers.ie/}, which allows small companies to
avail of Irish research facilities. Performed a feasibility study of the
application of ontologies in their software model.\\

Java consultancy for UCD Sport Science department. Worked on
improving an existing posture monitoring system. Integrated robust bluetooth 
connectivity protocols, improved data processing algorithms and
improved the viewer panel.

%\subsection*{Reviewing}
%\textit{PerCom '09} 8th Annual IEEE International Conference on
%Pervasive Computing and Communications\\
%
%\textit{SEACUBE '09} 1st international conference on Sensing and Acting
%in Ubiquitous Environments. On program committee of SEACUBE 2010.\\
%
%\textit{Pervasive '08}  6th International Conference on Pervasive
%Computing\\
%
%\textit{Revue d'intelligence Artificielle journal 2008}. Special
%issue on Modelling and Reasoning on Context: The Role of Context in Human
%Tasks.\\
%
%\textit{UbiComp '07}. 9th International Conference on Ubiquitous
%Computing.

%\subsection*{Conferences attended}
%\textit{FLAIRS '10} The 23rd International FLAIRS conference.\\
%
%\textit{PerCom '09}. 7th Annual IEEE International Conference on
%Pervasive Computing and Communications.\\
%
%\textit{ICPS '08} International Conference on Pervasive Services. (Presented)\\
%
%\textit{CONTEXT '07}. 6th International and Interdisciplinary Conference on Modeling and Using
%Context.\\
%
%\textit{Pervasive '06}. The The 4th International Conference on
%Pervasive Computing.\\
%
%\textit{PerCom '06}. The 4th Annual IEEE International Conference on
%Pervasive Computing and Communications.
%
%\subsection*{Student volunteer}
%\textit{Pervasive '06}. The The 4th International Conference on
%Pervasive Computing\\
%\textit{ICAC '06}. The The 3rd International Conference on
%Autonomic Computing

\section{Other Interests}
I am a very active sports person with an interest in motorsport and snowboarding
and keeping fit by running regularly. I also organise a weekly 5-a-side football game.
I practiced Karate for over ten years, Representing Ireland in
international competitions, as well as representing DIT and TCD in intervarsity competitions.
Received the grade of Nidan(second degree black belt) in November 2009.
	
%\section{Selected Publications}
%\nocite{knox2015egu}
%\nocite{knox2015pynsim}
%\nocite{knox2015hydra}
%\nocite{knox2010cbr}
%\nocite{yoon2015jordan}
%\nocite{Stevenson2009ontonym}
%\nocite{ScatterboxRIA2008}
%\nocite{Knox:2007:Towards-Scatter}
%\nocite{Dobson:2007:A-first-approac}
%\nocite{knox2014iemss}
%\nocite{meier2014iemss}
%\nocite{konur2014formal}
%\nocite{Coyle2007Proposed}
%\nocite{Clear2006Integrating}

%\bibliographystyle{plain}
%\bibliography{knox}

\section{Referees}
Available on request
%\textbf{Prof. Simon Dobson}, 
%Jack Cole Building, Room 1.17, University of St Andrews, North Haugh, St
%Andrews, Fife KY16 9SX, UK\\ \textbf{Phone:} +44 1334 461626\\
%
%\textbf{Will Slater}, Geneity Ltd, Fisher's Lane, Chiswick, London W138QN.\\
%\textbf{Phone:} +44 (0)208 742 8393

\end{document}

%%%%%%%%%%%%%%%%%%%%%%%%%% End CV Document %%%%%%%%%%%%%%%%%%%%%%%%%%%%%
